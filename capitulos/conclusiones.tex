\chapter{Conclusiones}

Al realizar estas prácticas de empresa no he necesitado mucho tiempo de formación más allá del que cualquier otra persona habría necesitado. Pues apenas he necesitado una formación extra en \textit{Ruby on Rails} y en \textit{CoffeeScript}, un \textit{framework} y un lenguaje específicos que se puede entender que no se vean durante nuestra formación académica.

Sin embargo, me parece grave que haya estudiantes que terminen su formación sin saber que existen los \textit{frameworks} de desarrollo web. En mi caso, aprendí \textit{Django} en una asignatura del segundo cuatrimestre del máster, pero muchos de los que acaban el grado, si no han escogido una optativa específica terminan sin conocer ningún \textit{framework}, y lo que es más grave, otros, dependiendo de la especialidad que hayan escogido, no han visto ni siquiera nada de desarrollo web el cuál es importantísimo hoy en día y, en mi opinión, debería conocer cualquier ingeniero informático.

Hablo de conocer un \textit{framework} porque pasar de uno a otro es trivial y solo hay que tener en cuenta las peculiaridades que tenga o el lenguaje que se utilice. Ya que la mayoría se basan en rutas, modelos, controladores y vistas.

Otro de los puntos importantes que he tenido que aprender en la empresa es el uso de \textit{Git} de forma colaborativa. En \textit{Cloud Computing: Fundamentos e Infraestructuras} del máster ha sido la primera vez que he tenido que utilizar \textit{Git} de forma obligatoria en una asignatura y aún así, aprendí un uso muy básico pues no era tampoco contenido implícito del temario. Sería interesante tener una asignatura, o varios seminarios y prácticas obligatorios en cursos avanzados del grado para aprender a utilizar un CVS (Sistema de Control de Versiones, del inglés \textit{Control Version System}) pues lo más seguro es que se necesite en un futuro empleo.