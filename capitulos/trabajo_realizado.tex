\chapter{Trabajo realizado}

Las prácticas de empresa del máster en ingeniería informática cubren un total de 150 horas. Lo que viene a ser seis semanas de lunes a viernes con una jornada laboral de cinco horas.

\section{Primera semana}

Durante la primera semana estuve de formación pues no conocía \textit{Ruby on Rails} que es el \textit{framework} que utilizan en la mayoría de los proyectos web.

El primer día estuve instalando y configurando las herramientas que iba a necesitar en mi ordenador y siguiendo rápidamente un curso \textit{online} gratuito de \textit{Ruby} para refrescar conocimientos, pues es un lenguaje que ya había utilizado en el segundo curso del grado en ingeniería informática en la asignatura de Programación y Diseño Orientado a Objetos. Además, sabiendo que era el que utilizaban, tres semanas antes de comenzar las prácticas pude aprovechar unas horas libres para hacer un curso en el que complementé los conocimientos que adquirí en la asignatura llegando a implementar en \textit{Ruby} un sistema de recomendaciones como el que realicé en \textit{Java} en la asignatura del máster de Gestión de Información en la Web. Este se encuentra disponible en mi cuenta de \textit{GitHub}: \href{https://github.com/fblupi/recommender-system-cf-ruby}{fblupi/recommender-system-cf-ruby}.

Como ya tenía conocimientos de \textit{Ruby} realicé muy rápido éste tutorial, lo que me permitió, a partir del segundo día, comenzar a hacer el curso, también \textit{online} y gratuito de \textit{Ruby on Rails} con el que se aprendía a crear un blog desde cero con base de datos, sesiones, etc.

Este curso me llevó el resto de días de la semana pues era bastante más extenso. Además de encontrarme con un fallo en mitad del desarrollo que no llegué a detectar y dejaba parte de la aplicación inutilizable (impedía llamar a métodos con el verbo \textit{HTTP} \textit{DELETE}) y no llegué a detectar, y al no haber utilizado un sistema de control de versiones, un error por mi parte, perdí cuatro horas en rehacer el trabajo. Fue entonces, al rehacer lo que llevaba de aplicación de nuevo, cuando detecté que era un problema por la incompatibilidad entre dos gemas que utilizaba.

El proyecto del blog realizado con este curso lo tengo disponible en mi cuenta de \textit{GitHub}: \href{https://github.com/fblupi/blogfacilito}{fblupi/blogfacilito}.

\section{Segunda semana}

Una vez terminado el primer curso de \textit{Ruby on Rails} empecé con otro, bastante más pequeño que me llevó media mañana, pero me permitió afianzar conocimientos. También era de realizar un blog y se encuentra en un repositorio de \textit{GitHub} \href{https://github.com/fblupi/blog-librosweb}{fblupi/blog-librosweb}.

Ya con conocimiento de \textit{Ruby on Rails} comencé a realizar tareas de administración en uno de los proyectos intercalando con un curso de \textit{CoffeeScript} cuyo contenido se puede ver en uno de mis repositorios de \textit{GitHub}: \href{https://github.com/fblupi/learning-coffeescript}{fblupi/learning-coffeescript}.

Para comprobar mis conocimientos de \textit{CoffeeScript} realicé el mismo sistema de recomendaciones que realicé en \textit{Ruby} en este otro lenguaje y también lo subí a \textit{GitHub}: \href{https://github.com/fblupi/recommender-system-cf-coffee}{fblupi/recommender-system-cf-coffee}

Las tareas de administración que realicé fueron las de migrar estaciones de una plataforma antigua a una nueva, pues algunas variables de algunos sensores necesitaban que se les aplicase una transformación para mostrar el valor en las unidades que los clientes quieren.

Esto me ayudó a, sin ver el código, tener una visión general del proyecto. Ya que desde el panel de administración de \textit{Rails}, donde realizaba todas las operaciones, podía ver la estructura de clases que se seguía.

\section{Tercera semana}

En esta tercera semana comencé con una tarea sencilla de desarrollo, además aprendí a utilizar \textit{RubyMine} ya que la empresa compra licencias de este IDE (Entorno de Desarrollo Integrado, del inglés \textit{Integrated Development Environment}) que cuenta con muchas funcionalidades que hacen más sencillo el seguimiento de la traza por donde se ejecuta el programa entre todos los distintos ficheros. Además de tener una integración total con \textit{Git}, lo que permite ver quién hizo qué cambio en cada línea para ir a preguntar directamente a ese compañero si surge alguna duda al respecto.

Mi objetivo con esta tarea era crear un nuevo tipo de estación, lo que permitiría especificar funcionalidades para éstas. Mis conocimientos en asignaturas como Programación y Diseño Orientado a Objetos, Sistemas de Información Basados en Web del grado o Sistemas Software Basados en Web del máster me fueron de mucha utilidad para llevar a cabo esta tarea.

