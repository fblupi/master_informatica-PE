\documentclass[a4paper,11pt]{book}
\usepackage{listings}
\usepackage[utf8]{inputenc}
\usepackage[spanish]{babel}

\decimalpoint
\usepackage{dcolumn}
\newcolumntype{.}{D{.}{\esperiod}{-1}}
\makeatletter
\addto\shorthandsspanish{\let\esperiod\es@period@code}
\makeatother

\RequirePackage{verbatim}
\usepackage{fancyhdr}
\usepackage{graphicx}
\usepackage{afterpage}

\usepackage{longtable}

\usepackage[pdfborder={000}]{hyperref} %referencia

% ********************************************************************
% Re-usable information
% ********************************************************************
\newcommand{\myDocument}{Memoria de prácticas}
\newcommand{\myTitle}{Prácticas de empresa}
\newcommand{\myCompany}{Bi 4 Group}
\newcommand{\myDegree}{Máster en Ingeniería Informática}
\newcommand{\myName}{Aythami Estévez Olivas}
\newcommand{\myProf}{Pedro Villar Castro (tutor académico)}
\newcommand{\myOtherProf}{Nombre Apllido1 Apellido2 (tutor en la empresa)}
\newcommand{\myFaculty}{Escuela Técnica Superior de Ingenierías Informática y de Telecomunicación}
\newcommand{\myFacultyShort}{E.T.S.I.I.T.}
\newcommand{\myUni}{\protect{Universidad de Granada}}
\newcommand{\myLocation}{Granada}
\newcommand{\myTime}{\today}


\hypersetup{
pdfauthor = {\myName (aythae@correo.ugr.es)},
pdftitle = {\myTitle},
pdfsubject = {},
pdfkeywords = {prácticas_empresa, memoria, bi4group},
pdfcreator = {LaTeX con el paquete MiKTeX},
pdfproducer = {pdflatex}
}

\usepackage{url}
\usepackage{colortbl,longtable}
\usepackage[stable]{footmisc}

\pagestyle{fancy}
\fancyhf{}
\fancyhead[LO]{\leftmark}
\fancyhead[RE]{\rightmark}
\fancyhead[RO,LE]{\textbf{\thepage}}
\renewcommand{\chaptermark}[1]{\markboth{\textbf{#1}}{}}
\renewcommand{\sectionmark}[1]{\markright{\textbf{\thesection. #1}}}

\setlength{\headheight}{1.5\headheight}

\newcommand{\HRule}{\rule{\linewidth}{0.5mm}}

\definecolor{gray97}{gray}{.97}
\definecolor{gray75}{gray}{.75}
\definecolor{gray45}{gray}{.45}
\definecolor{gray30}{gray}{.94}

\lstset{ frame=Ltb,
     framerule=0.5pt,
     aboveskip=0.5cm,
     framextopmargin=3pt,
     framexbottommargin=3pt,
     framexleftmargin=0.1cm,
     framesep=0pt,
     rulesep=.4pt,
     backgroundcolor=\color{gray97},
     rulesepcolor=\color{black},
     %
     stringstyle=\ttfamily,
     showstringspaces = false,
     basicstyle=\scriptsize\ttfamily,
     commentstyle=\color{gray45},
     keywordstyle=\bfseries,
     %
     numbers=left,
     numbersep=6pt,
     numberstyle=\tiny,
     numberfirstline = false,
     breaklines=true,
   }
 
% minimizar fragmentado de listados
\lstnewenvironment{listing}[1][]
   {\lstset{#1}\pagebreak[0]}{\pagebreak[0]}

\lstdefinestyle{CodigoC}
   {
	basicstyle=\scriptsize,
	frame=single,
	language=C,
	numbers=left
   }

\lstdefinestyle{CodigoC++}
   {
	basicstyle=\small,
	frame=single,
	backgroundcolor=\color{gray30},
	language=C++,
	numbers=left
   }
 
\lstdefinestyle{Consola}
   {basicstyle=\scriptsize\bf\ttfamily,
    backgroundcolor=\color{gray30},
    frame=single,
    numbers=none
   }

\newcommand{\bigrule}{\titlerule[0.5mm]}


%Para conseguir que en las páginas en blanco no ponga cabecerass
\makeatletter
\def\clearpage{%
  \ifvmode
    \ifnum \@dbltopnum =\m@ne
      \ifdim \pagetotal <\topskip
        \hbox{}
      \fi
    \fi
  \fi
  \newpage
  \thispagestyle{empty}
  \write\m@ne{}
  \vbox{}
  \penalty -\@Mi
}
\makeatother

\usepackage{pdfpages}
\begin{document}

\begin{titlepage}
 
 
\newlength{\centeroffset}
\setlength{\centeroffset}{-0.5\oddsidemargin}
\addtolength{\centeroffset}{0.5\evensidemargin}
\thispagestyle{empty}

\noindent\hspace*{\centeroffset}

\begin{minipage}{\textwidth}

\centering
\includegraphics[width=0.9\textwidth]{imagenes/logo_ugr.jpg}\\[1.4cm]

\textsc{ \Large \MakeUppercase{\myDocument}\\[0.2cm]}
\textsc{ \MakeUppercase{\myDegree}}\\[1cm]
% Upper part of the page
% 
% Title
{\Huge\bfseries \myTitle \\
}
\noindent\rule[-1ex]{\textwidth}{3pt}\\[3.5ex]
{\large\bfseries \myCompany}\\[0.6cm]
\includegraphics[width=0.5\textwidth]{imagenes/bi4group.png}\\

\end{minipage}

\vspace{1cm}
\noindent\hspace*{\centeroffset}

\begin{minipage}{\textwidth}
\centering

\textbf{Autor}\\ {\myName}\\[2.5ex]
\textbf{Directores}\\
{\myProf\\
\myOtherProf}\\[1cm]
\includegraphics[width=0.3\textwidth]{imagenes/etsiit_logo.png}\\[0.1cm]
\textsc{\myFacultyShort}\\
\textsc{---}\\
\myLocation, \myTime
\end{minipage}
%\addtolength{\textwidth}{\centeroffset}
%\vspace{\stretch{2}}
\end{titlepage}



\frontmatter
\tableofcontents

\mainmatter
\setlength{\parskip}{5pt}
\chapter{Introducción}

En este documento voy a presentar mi memoria de prácticas de empresa del máster en ingeniería informática de la Universidad de Granada.

Comenzaré con una sección en la que realizaré una descripción de la empresa en la que he realizado tales prácticas, Nazaríes IT.

En la siguiente sección contaré el trabajo que he ido realizando durante las seis semanas indicando las tareas, así como las soluciones aportadas, las herramientas utilizadas y las asignaturas de la titulación que han resultado útiles para resolverlas.

A continuación realizaré una valoración personal del trabajo que he realizado y acabaré con unas conclusiones.
\chapter{Descripción de la entidad donde realizó las prácticas}

Bi4 Group se define como una empresa que desarrolla soluciones software a medida para sus clientes \cite{bi4_about_us}. Nació en Ámsterdam en 2012 y actualmente cuenta con unos 30 empleados repartidos entre sus oficinas de Ámsterdam, Madrid y Granada. Aunque la mayor parte de los empleados se concentran en Granada por ser donde se encuentra todo el desarrollo, el resto de oficinas cuentan con personal administrativo o de ventas además de los posibles desarrolladores desplazados al lugar para ponerse en contacto con los clientes. 

Su principal área de negocio es la creación de aplicaciones web complejas, incluyendo elementos de \textit{Business Intelligence}, visualización de datos y \textit{Big Data} siguiendo un modelo de \textit{Custom Software as a Service (CSaaS)} que se vale del desarrollo basado en componentes y despliegues en la nube para agilizar el proceso de desarrollo, reutilizando los componentes comunes de las aplicaciones y centrándose en adaptar el producto a las necesidades específicas del cliente.

Como metodología de desarrollo emplean múltiples métodos ágiles como Scrum y Kanban donde el cliente siempre está presente de modo que la comunicación con él es continua y siempre conoce el estado del proyecto. Estos métodos utilizan pequeños equipos de 5-6 personas máximo, que se autoorganizan y permiten llevar a cabo múltiples proyectos en paralelo aún con un pequeño número de empleados.


\chapter{Trabajo realizado}

Las prácticas de empresa del máster en ingeniería informática cubren un total de 150 horas. Lo que viene a ser seis semanas de lunes a viernes con una jornada laboral de cinco horas.

\section{Primera semana}

Durante la primera semana estuve de formación pues no conocía \textit{Ruby on Rails} que es el \textit{framework} que utilizan en la mayoría de los proyectos web.

El primer día estuve instalando y configurando las herramientas que iba a necesitar en mi ordenador y siguiendo rápidamente un curso \textit{online} gratuito de \textit{Ruby} para refrescar conocimientos, pues es un lenguaje que ya había utilizado en el segundo curso del grado en ingeniería informática en la asignatura de Programación y Diseño Orientado a Objetos. Además, sabiendo que era el que utilizaban, tres semanas antes de comenzar las prácticas pude aprovechar unas horas libres para hacer un curso en el que complementé los conocimientos que adquirí en la asignatura llegando a implementar en \textit{Ruby} un sistema de recomendaciones como el que realicé en \textit{Java} en la asignatura del máster de Gestión de Información en la Web. Este se encuentra disponible en mi cuenta de \textit{GitHub}: \url{https://github.com/fblupi/recommender-system-cf-ruby}

Como ya tenía conocimientos de \textit{Ruby} realicé muy rápido éste tutorial, lo que me permitió, a partir del segundo día, comenzar a hacer el curso, también \textit{online} y gratuito de \textit{Ruby on Rails} con el que se aprendía a crear un blog desde cero con base de datos, sesiones, etc.

Este curso me llevó el resto de días de la semana pues era bastante más extenso. Además de encontrarme con un fallo en mitad del desarrollo que no llegué a detectar y dejaba parte de la aplicación inutilizable (impedía llamar a métodos con el verbo \textit{HTTP} \textit{DELETE}) y no llegué a detectar, y al no haber utilizado un sistema de control de versiones, un error por mi parte, perdí cuatro horas en rehacer el trabajo. Fue entonces, al rehacer lo que llevaba de aplicación de nuevo, cuando detecté que era un problema por la incompatibilidad entre dos gemas que utilizaba.

El proyecto del blog realizado con este curso lo tengo disponible en mi cuenta de \textit{GitHub}: \url{https://github.com/fblupi/blogfacilito}

\chapter{Valoración personal del trabajo realizado}

Lala
\chapter{Conclusiones}

Al realizar estas prácticas de empresa no he necesitado mucho tiempo de formación más allá del que cualquier otra persona habría necesitado. Pues apenas he necesitado una formación extra en \textit{Ruby on Rails} y en \textit{CoffeeScript}, un \textit{framework} y un lenguaje específicos que se puede entender que no se vean durante nuestra formación académica.

Sin embargo, me parece grave que haya estudiantes que terminen su formación sin saber que existen los \textit{frameworks} de desarrollo web. En mi caso, aprendí \textit{Django} en una asignatura del segundo cuatrimestre del máster, pero muchos de los que acaban el grado, si no han escogido una optativa específica terminan sin conocer ningún \textit{framework}, y lo que es más grave, otros, dependiendo de la especialidad que hayan escogido, no han visto ni siquiera nada de desarrollo web el cuál es importantísimo hoy en día y, en mi opinión, debería conocer cualquier ingeniero informático.

Hablo de conocer un \textit{framework} porque pasar de uno a otro es trivial y solo hay que tener en cuenta las peculiaridades que tenga o el lenguaje que se utilice. Ya que la mayoría se basan en rutas, modelos, controladores y vistas.

Otro de los puntos importantes que he tenido que aprender en la empresa es el uso de \textit{Git} de forma colaborativa. En \textit{Cloud Computing: Fundamentos e Infraestructuras} del máster ha sido la primera vez que he tenido que utilizar \textit{Git} de forma obligatoria en una asignatura y aún así, aprendí un uso muy básico pues no era tampoco contenido implícito del temario. Sería interesante tener una asignatura, o varios seminarios y prácticas obligatorios en cursos avanzados del grado para aprender a utilizar un CVS (Sistema de Control de Versiones, del inglés \textit{Control Version System}) pues lo más seguro es que se necesite en un futuro empleo.

\bibliography{bibliografia/bibliografia}
\addcontentsline{toc}{chapter}{Bibliografía}
\bibliographystyle{plain}

\end{document}
